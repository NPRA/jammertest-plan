% Content below is autogenerated 
\chapter{0 Grace period}

\section{0.1: Grace period}

\subsection*{Rationale}

In order for equipment to return to normal operation after interference, a grace period is provided between tests.

\subsection*{Test description}

This period can be used to make sure that equipment is ready for upcoming tests.

\section*{Test within this testgroup}

\subsection{0.1.1  Grace period}

\textcolor{lightgray}{\noindent\rule[0.5ex]{\linewidth}{1pt} }
No RF interference expected in this test.
\subsubsection*{Power or power range}
Min: 0W\\Max: 0W\subsubsection*{Test bands/constellation}
'N/A'
\subsubsection*{Transmitter equpment}
'N/A'
\\\chapter{1 jamming}

\section{1.1: Continuous stationary low power jamming with commercially available jammers}

\subsection*{Rationale}

The main objective is to observe how the J/S signal affect the availability of PNT, and/or how it produces inaccurate PNT data, when the jamming signal (J) is generated by low-power jammers commercially available online.

\subsection*{Additional information}

Spesification of jammers can be found in appendix A

\section*{Test within this testgroup}

\subsection{1.1.1  Jammer S1.1}

\textcolor{lightgray}{\noindent\rule[0.5ex]{\linewidth}{1pt} }
Test with jammer S1.1
\subsubsection*{Power or power range}
Min: 0.01W\\Max: 0.0316W\subsubsection*{Test bands/constellation}
'L1', ' E1', ' B1l', ' B1C'
\subsubsection*{Transmitter equpment}
'S1.1'
\\\subsection{1.1.2  Jammer S1.2}

\textcolor{lightgray}{\noindent\rule[0.5ex]{\linewidth}{1pt} }
Test with jammer S1.2
\subsubsection*{Power or power range}
Min: 0.01W\\Max: 0.0316W\subsubsection*{Test bands/constellation}
'L1', ' E1', ' B1l', ' B1C'
\subsubsection*{Transmitter equpment}
'S1.2'
\\\subsection{1.1.3  Jammer S1.3}

\textcolor{lightgray}{\noindent\rule[0.5ex]{\linewidth}{1pt} }
Test with jammer S1.3
\subsubsection*{Power or power range}
Min: 0.01W\\Max: 0.0316W\subsubsection*{Test bands/constellation}
'L1', ' E1', ' B1l', ' B1C'
\subsubsection*{Transmitter equpment}
'S1.3'
\\\subsection{1.1.4  Jammer S2.1}

\textcolor{lightgray}{\noindent\rule[0.5ex]{\linewidth}{1pt} }
Test with jammer S2.1
\subsubsection*{Power or power range}
Min: 0.0316W\\Max: 0.1W\subsubsection*{Test bands/constellation}
'L1', ' E1', ' B1l', ' B1C', ' L5', ' E5a/b', ' B2a/b', ' G3'
\subsubsection*{Transmitter equpment}
'S2.1'
\\\subsection{1.1.5  Jammer S2.2}

\textcolor{lightgray}{\noindent\rule[0.5ex]{\linewidth}{1pt} }
Test with jammer S2.2
\subsubsection*{Power or power range}
Min: 0.0316W\\Max: 0.1W\subsubsection*{Test bands/constellation}
'L1', ' E1', ' B1l', ' B1C', ' L5', ' E5a/b', ' B2a/b', ' G3'
\subsubsection*{Transmitter equpment}
'S2.2'
\\\subsection{1.1.6  Jammer S2.3}

\textcolor{lightgray}{\noindent\rule[0.5ex]{\linewidth}{1pt} }
Test with jammer S2.3
\subsubsection*{Power or power range}
Min: 0.0316W\\Max: 0.1W\subsubsection*{Test bands/constellation}
'L1', ' E1', ' B1l', ' B1C', ' L5', ' E5a/b', ' B2a/b', ' G3'
\subsubsection*{Transmitter equpment}
'S2.3'
\\\subsection{1.1.7  Jammer S2.4}

\textcolor{lightgray}{\noindent\rule[0.5ex]{\linewidth}{1pt} }
Test with jammer S2.4
\subsubsection*{Power or power range}
Min: 0.0316W\\Max: 0.1W\subsubsection*{Test bands/constellation}
'L1', ' E1', ' B1l', ' B1C', ' L5', ' E5a/b', ' B2a/b', ' G3'
\subsubsection*{Transmitter equpment}
'S2.4'
\\\subsection{1.1.8  Jammer U1.1}

\textcolor{lightgray}{\noindent\rule[0.5ex]{\linewidth}{1pt} }
Test with jammer U1.1
\subsubsection*{Power or power range}
Min: 0W\\Max: 0W\subsubsection*{Test bands/constellation}
'L1', ' E1', ' B1l', ' B1C', ' G1'
\subsubsection*{Transmitter equpment}
'U1.1'
\\\subsection{1.1.9  Jammer U1.2}

\textcolor{lightgray}{\noindent\rule[0.5ex]{\linewidth}{1pt} }
Test with jammer U1.2
\subsubsection*{Power or power range}
Min: 0W\\Max: 0W\subsubsection*{Test bands/constellation}
'L1', ' E1', ' B1l', ' B1C', ' G1'
\subsubsection*{Transmitter equpment}
'U1.2'
\\\subsection{1.1.10  Jammer U1.3}

\textcolor{lightgray}{\noindent\rule[0.5ex]{\linewidth}{1pt} }
Test with jammer U1.3
\subsubsection*{Power or power range}
Min: 0W\\Max: 0W\subsubsection*{Test bands/constellation}
'L1', ' E1', ' B1l', ' B1C', ' G1'
\subsubsection*{Transmitter equpment}
'U1.3'
\\\subsection{1.1.11  Jammer U1.4}

\textcolor{lightgray}{\noindent\rule[0.5ex]{\linewidth}{1pt} }
Test with jammer U1.4
\subsubsection*{Power or power range}
Min: 0W\\Max: 0W\subsubsection*{Test bands/constellation}
'L1', ' E1', ' B1l', ' B1C', ' G1'
\subsubsection*{Transmitter equpment}
'U1.4'
\\\subsection{1.1.12  Jammer H1.1}

\textcolor{lightgray}{\noindent\rule[0.5ex]{\linewidth}{1pt} }
Jammer H1.1 - high power, GPS L1+L2, wideband modulation. Will be activated in high power mode, for GPS L1 and L2 with modulation set for wideband.
\subsubsection*{Power or power range}
Min: 0.0003W\\Max: 0.1W\subsubsection*{Test bands/constellation}
'L1', ' L2'
\subsubsection*{Transmitter equpment}
'H1.1'
\\\subsection{1.1.13  Jammer H1.2}

\textcolor{lightgray}{\noindent\rule[0.5ex]{\linewidth}{1pt} }
Jammer H1.2
\subsubsection*{Power or power range}
Min: 0.0631W\\Max: 0.0631W\subsubsection*{Test bands/constellation}
'L1', ' E1', ' B1C'
\subsubsection*{Transmitter equpment}
'H1.2'
\\\subsection{1.1.14  Jammer H3.1}

\textcolor{lightgray}{\noindent\rule[0.5ex]{\linewidth}{1pt} }
Jammer H3.1
\subsubsection*{Power or power range}
Min: 0.1W\\Max: 0.1W\subsubsection*{Test bands/constellation}
'L1', ' E1', ' B1C', ' B1l'
\subsubsection*{Transmitter equpment}
'H3.1'
\\\subsection{1.1.15  Jammer H3.2}

\textcolor{lightgray}{\noindent\rule[0.5ex]{\linewidth}{1pt} }
Jammer H3.2
\subsubsection*{Power or power range}
Min: 0.1W\\Max: 0.1W\subsubsection*{Test bands/constellation}
'L1', ' E1', ' B1C', ' B1l'
\subsubsection*{Transmitter equpment}
'H3.2'
\\\subsection{1.1.16  Jammer H3.3}

\textcolor{lightgray}{\noindent\rule[0.5ex]{\linewidth}{1pt} }
Jammer H3.3
\subsubsection*{Power or power range}
Min: 1W\\Max: 1W\subsubsection*{Test bands/constellation}
'L1', ' E1', ' B1C', ' L2', ' L5', ' E5a', ' B2a'
\subsubsection*{Transmitter equpment}
'H3.3'
\\\subsection{1.1.17  Jammer H4.1}

\textcolor{lightgray}{\noindent\rule[0.5ex]{\linewidth}{1pt} }
Jammer H4.1
\subsubsection*{Power or power range}
Min: 0.3981W\\Max: 0.631W\subsubsection*{Test bands/constellation}
'L1', ' E1', ' B1C', ' B1l', ' E6', ' G2', ' B3l', ' L2', ' G2', ' B2b', ' E5b', ' L5', ' G3', ' B2a', ' E5a/b'
\subsubsection*{Transmitter equpment}
'H4.1'
\\\subsection{1.1.18  Jammer H6.1}

\textcolor{lightgray}{\noindent\rule[0.5ex]{\linewidth}{1pt} }
Jammer H6.1
\subsubsection*{Power or power range}
Min: 0.631W\\Max: 0.631W\subsubsection*{Test bands/constellation}
'L1', ' E1', ' B1C'
\subsubsection*{Transmitter equpment}
'H6.1'
\\\subsection{1.1.19  Jammer H6.2}

\textcolor{lightgray}{\noindent\rule[0.5ex]{\linewidth}{1pt} }
Jammer H6.2
\subsubsection*{Power or power range}
Min: 0.3981W\\Max: 1W\subsubsection*{Test bands/constellation}
'L1', ' E1', ' B1C', ' L5', ' G3', ' B2a/b', ' E5a/b', ' L2', ' G2', ' G3', ' B2b', ' B3l', ' E5b', ' E6'
\subsubsection*{Transmitter equpment}
'H6.2'
\\\subsection{1.1.20  Jammer H6.3}

\textcolor{lightgray}{\noindent\rule[0.5ex]{\linewidth}{1pt} }
Jammer H6.3
\subsubsection*{Power or power range}
Min: 0.3981W\\Max: 1W\subsubsection*{Test bands/constellation}
'L1', ' E1', ' B1C', ' L5', ' G3', ' B2a/b', ' E5a/b', ' L2', ' G2', ' G3', ' B2b', ' B3l', ' E5b', ' E6'
\subsubsection*{Transmitter equpment}
'H6.3'
\\\subsection{1.1.21  Jammer H6.4}

\textcolor{lightgray}{\noindent\rule[0.5ex]{\linewidth}{1pt} }
Jammer H6.4
\subsubsection*{Power or power range}
Min: 1W\\Max: 1.58W\subsubsection*{Test bands/constellation}
'L5', ' B2a', ' E5a', ' L2', ' G2', ' B3l', ' E6', ' L1', ' E1', ' B1C', ' B1l', ' G1'
\subsubsection*{Transmitter equpment}
'H6.4'
\\\subsection{1.1.22  Jammer H6.5}

\textcolor{lightgray}{\noindent\rule[0.5ex]{\linewidth}{1pt} }
Jammer H6.5
\subsubsection*{Power or power range}
Min: 1W\\Max: 1.58W\subsubsection*{Test bands/constellation}
'L5', ' B2a', ' E5a', ' L2', ' G2', ' B3l', ' E6', ' L1', ' E1', ' B1C', ' B1l', ' G1'
\subsubsection*{Transmitter equpment}
'H6.5'
\\\subsection{1.1.23  Jammer H6.6}

\textcolor{lightgray}{\noindent\rule[0.5ex]{\linewidth}{1pt} }
Jammer H6.6
\subsubsection*{Power or power range}
Min: 1W\\Max: 1.58W\subsubsection*{Test bands/constellation}
'L5', ' B2a', ' E5a', ' L2', ' G2', ' B3l', ' E6', ' L1', ' E1', ' B1C', ' B1l', ' G1'
\subsubsection*{Transmitter equpment}
'H6.6'
\\\subsection{1.1.24  Jammer H8.1}

\textcolor{lightgray}{\noindent\rule[0.5ex]{\linewidth}{1pt} }
Jammer H8.1
\subsubsection*{Power or power range}
Min: 0.631W\\Max: 0.631W\subsubsection*{Test bands/constellation}
'L1', ' E1', ' B1C', ' B1l', ' G1'
\subsubsection*{Transmitter equpment}
'H8.1'
\\\subsection{1.1.25  Jammer F6.1}

\textcolor{lightgray}{\noindent\rule[0.5ex]{\linewidth}{1pt} }
Jammer F6.1 - Full power antenna F2 to F6
\subsubsection*{Power or power range}
Min: 0.5012W\\Max: 6.31W\subsubsection*{Test bands/constellation}
'L1', ' E1', ' B1C', ' B1l', ' G1', ' L2', ' G2', ' B3l', ' B2b', ' E6', ' L5', ' E5a', ' B2a'
\subsubsection*{Transmitter equpment}
'F6.1'
\\\subsection{1.1.26  Jammer H1.3}

\textcolor{lightgray}{\noindent\rule[0.5ex]{\linewidth}{1pt} }
Jammer H1.3
\subsubsection*{Power or power range}
Min: 0W\\Max: 0W\subsubsection*{Test bands/constellation}
'L1', ' E1', ' B1C'
\subsubsection*{Transmitter equpment}
'H1.3'
\\\subsection{1.1.27  Jammer H2.1}

\textcolor{lightgray}{\noindent\rule[0.5ex]{\linewidth}{1pt} }
Jammer H2.1
\subsubsection*{Power or power range}
Min: 0W\\Max: 0W\subsubsection*{Test bands/constellation}
'L1', ' E1', ' B1C', ' L2'
\subsubsection*{Transmitter equpment}
'H2.1'
\\\subsection{1.1.28  Jammer H2.2}

\textcolor{lightgray}{\noindent\rule[0.5ex]{\linewidth}{1pt} }
Jammer H2.2
\subsubsection*{Power or power range}
Min: 0W\\Max: 0W\subsubsection*{Test bands/constellation}
'L1', ' E1', ' B1C', ' L2'
\subsubsection*{Transmitter equpment}
'H2.2'
\\\section{1.2: Continuous stationary high-power jamming with CW}

\subsection*{Rationale}

The main objective is to observe how the Jammer signal to GNSS signal (J/S) ratio affect the availability of PNT, and/or how it produces inaccurate PNT data.

\subsection*{Test description}

The use of continuous high-power jamming will block GNSS signals in a large area at the event. The attendees may therefore test their equipment at different ranges to such a high-power jammer. There will be transmitted with a continuous wave (CW) modulation (single frequency component) using Right Hand Circular Polarized (RHCP) antennas. The use of a 20W jammer will result in among the highest J/S ratios during the event. The attendees can change distance to the transmitter and observe the changes and try to identify the protection ratio for their GNSS receiving system.

\subsection*{Additional information}

The jammer employed will be F8.1 "Porcus Major", see appendix A.

\section*{Test within this testgroup}

\subsection{1.2.1  20W CW: L1}

\textcolor{lightgray}{\noindent\rule[0.5ex]{\linewidth}{1pt} }
20W CW: L1
\subsubsection*{Power or power range}
Min: 20W\\Max: 20W\subsubsection*{Test bands/constellation}
'L1'
\subsubsection*{Transmitter equpment}
'F8.1'
\\\subsection{1.2.2  20W CW: L1, G1}

\textcolor{lightgray}{\noindent\rule[0.5ex]{\linewidth}{1pt} }
20W CW: L1, G1
\subsubsection*{Power or power range}
Min: 20W\\Max: 20W\subsubsection*{Test bands/constellation}
'L1', ' G1'
\subsubsection*{Transmitter equpment}
'F8.1'
\\\subsection{1.2.3  20W CW: L1, G1, L2}

\textcolor{lightgray}{\noindent\rule[0.5ex]{\linewidth}{1pt} }
20W CW: L1, G1, L2
\subsubsection*{Power or power range}
Min: 20W\\Max: 20W\subsubsection*{Test bands/constellation}
'L1', ' G1', ' L2'
\subsubsection*{Transmitter equpment}
'F8.1'
\\\subsection{1.2.4  20W CW: L1, G1, L2, L5}

\textcolor{lightgray}{\noindent\rule[0.5ex]{\linewidth}{1pt} }
20W CW: L1, G1, L2, L5
\subsubsection*{Power or power range}
Min: 20W\\Max: 20W\subsubsection*{Test bands/constellation}
'L1', ' G1', ' L2', ' L5'
\subsubsection*{Transmitter equpment}
'F8.1'
\\\section{1.3: Continuous stationary high-power jamming with sweep/chirp}

\subsection*{Rationale}

The main objective is to observe how the Jammer signal to GNSS signal (J/S) ratio affect the availability of PNT, and/or how it produces inaccurate PNT data.

\subsection*{Test description}

The use of continuous high-power jamming will block GNSS signals in a large area at the event. The attendees may therefore test their equipment at different ranges to such a high-power jammer. There will be transmitted with a sweep/chirp modulation using Right Hand Circular Polarized (RHCP) antennas. Sweep/chirp modulation means that the frequency component will sweep back and forth inside the specific frequency band with a given sweep rate. The use of a 20W jammer will result in among the highest J/S ratios during the event. The attendees can change distance to the transmitter and observe the changes and try to identify the protection ratio for your GNSS receiving system.

\subsection*{Additional information}

The jammer employed will be F8.1 "Porcus Major", see appendix A.

\section*{Test within this testgroup}

\subsection{1.3.1  20W chirp: L1}

\textcolor{lightgray}{\noindent\rule[0.5ex]{\linewidth}{1pt} }
20W chirp: L1
\subsubsection*{Power or power range}
Min: 20W\\Max: 20W\subsubsection*{Test bands/constellation}
'L1'
\subsubsection*{Transmitter equpment}
'F8.1'
\\\subsection{1.3.2  20W chirp: L1, G1}

\textcolor{lightgray}{\noindent\rule[0.5ex]{\linewidth}{1pt} }
20W chirp: L1, G1
\subsubsection*{Power or power range}
Min: 20W\\Max: 20W\subsubsection*{Test bands/constellation}
'L1', ' G1'
\subsubsection*{Transmitter equpment}
'F8.1'
\\\subsection{1.3.3  20W chirp: L1, G1, L2}

\textcolor{lightgray}{\noindent\rule[0.5ex]{\linewidth}{1pt} }
20W chirp: L1, G1, L2
\subsubsection*{Power or power range}
Min: 20W\\Max: 20W\subsubsection*{Test bands/constellation}
'L1', ' G1', ' L2'
\subsubsection*{Transmitter equpment}
'F8.1'
\\\subsection{1.3.4  20W chirp: L1, G1, L2, L5}

\textcolor{lightgray}{\noindent\rule[0.5ex]{\linewidth}{1pt} }
20W chirp: L1, G1, L2, L5
\subsubsection*{Power or power range}
Min: 20W\\Max: 20W\subsubsection*{Test bands/constellation}
'L1', ' G1', ' L2', ' L5'
\subsubsection*{Transmitter equpment}
'F8.1'
\\\section{1.4: Continuous stationary high-power jamming with PRN}

\subsection*{Rationale}

The main objective is to observe how the Jammer signal to GNSS signal (J/S) ratio affect the availability of PNT, and/or how it produces inaccurate PNT data.

\subsection*{Test description}

The use of continuous high-power jamming will block out a large area at the event. The attendees may therefore test the range of such a high-power jammer. There will be transmitted with a Pseudo Random Noise (PRN) modulation using Right Hand Circular Polarized (RHCP) antennas. PRN signals have the same spectral form as the true signals sent from the GNSS satellites but with different spreading codes. The spreading codes are Binary Phase Shift Keying (BPSK) modulated onto the centre frequency of the specific GNSS band. The use of a 20W jammer will result in among the highest J/S ratios during the event. The attendees can change distance to the transmitter and observe the changes and try to identify the protection ratio for your GNSS receiving system.

\subsection*{Additional information}

The jammer employed will be F8.1 "Porcus Major", see appendix A.

\section*{Test within this testgroup}

\subsection{1.4.1  20W PRN: L1}

\textcolor{lightgray}{\noindent\rule[0.5ex]{\linewidth}{1pt} }
20W PRN: L1
\subsubsection*{Power or power range}
Min: 20W\\Max: 20W\subsubsection*{Test bands/constellation}
'L1'
\subsubsection*{Transmitter equpment}
'F8.1'
\\\subsection{1.4.2  20W PRN: L1, G1}

\textcolor{lightgray}{\noindent\rule[0.5ex]{\linewidth}{1pt} }
20W PRN: L1, G1
\subsubsection*{Power or power range}
Min: 20W\\Max: 20W\subsubsection*{Test bands/constellation}
'L1', ' G1'
\subsubsection*{Transmitter equpment}
'F8.1'
\\\subsection{1.4.3  20W PRN: L1, G1, L2}

\textcolor{lightgray}{\noindent\rule[0.5ex]{\linewidth}{1pt} }
20W PRN: L1, G1, L2
\subsubsection*{Power or power range}
Min: 20W\\Max: 20W\subsubsection*{Test bands/constellation}
'L1', ' G1', ' L2'
\subsubsection*{Transmitter equpment}
'F8.1'
\\\subsection{1.4.4  20W PRN: L1, G1, L2, L5}

\textcolor{lightgray}{\noindent\rule[0.5ex]{\linewidth}{1pt} }
20W PRN: L1, G1, L2, L5
\subsubsection*{Power or power range}
Min: 20W\\Max: 20W\subsubsection*{Test bands/constellation}
'L1', ' G1', ' L2', ' L5'
\subsubsection*{Transmitter equpment}
'F8.1'
\\\section{1.5: Continuous stationary high-power jamming with "real world" PRN}

\subsection*{Rationale}

The type of jamming employed in this test is the same as real world signals observed in Europe, where the jammer parameters were found after demodulating a captured baseband stream.

\subsection*{Test description}

The tests will be performed with BPSK modulation with a pseudo random symbol rate of 3 Mbaud at GPS L1 and 10.23 Mbaud at Glonass G1. The test cases refer to which centre frequency the signal will be centred at, based on the named GNSS bands.

\subsection*{Additional information}

The jammer employed will be F8.1 "Porcus Major", see appendix A.

\section*{Test within this testgroup}

\subsection{1.5.1  20W: L1, PRN (BPSK-modulated with 3 Mbaud symbolrate)}

\textcolor{lightgray}{\noindent\rule[0.5ex]{\linewidth}{1pt} }
20W: L1, PRN (BPSK-modulated with 3 Mbaud symbolrate)
\subsubsection*{Power or power range}
Min: 20W\\Max: 20W\subsubsection*{Test bands/constellation}
'L1'
\subsubsection*{Transmitter equpment}
'F8.1'
\\\subsection{1.5.2  20W: G1, PRN (BPSK-modulated with 10 Mbaud symbolrate)}

\textcolor{lightgray}{\noindent\rule[0.5ex]{\linewidth}{1pt} }
20W: G1, PRN (BPSK-modulated with 10 Mbaud symbolrate)
\subsubsection*{Power or power range}
Min: 20W\\Max: 20W\subsubsection*{Test bands/constellation}
'G1'
\subsubsection*{Transmitter equpment}
'F8.1'
\\\section{1.6: Stationary high-power jamming, ramp power with PRN}

\subsection*{Rationale}

The main objective is to observe how the J/S signal affect the loss of PNT, and/or how it produces inaccurate PNT data, and at which power level. This will allow for evaluation of the sensitivity thresholds for various systems.

\subsection*{Test description}

The attendees should be at a stationary location with a known distance to the jammer, so they can observe how different levels will affect the PNT. Comparing the ramping tests from different sites, will give the opportunity to compare signals arriving from different angles and also to see the difference between signals going along earth/ground and coming from above.

\subsection*{Additional information}

The jammer employed will be F8.1 "Porcus Major", see appendix A.

\section*{Test within this testgroup}

\subsection{1.6.1  0.1µW to 20W, 2 dB increments PRN: L1}

\textcolor{lightgray}{\noindent\rule[0.5ex]{\linewidth}{1pt} }
0.1µW to 20W, 2 dB increments PRN: L1
\subsubsection*{Power or power range}
Min: 1e-07W\\Max: 20W\subsubsection*{Test bands/constellation}
'L1'
\subsubsection*{Transmitter equpment}
'F8.1'
\\\subsection{1.6.2  0.1µW to 20W, 2 dB increments PRN: L1, G1}

\textcolor{lightgray}{\noindent\rule[0.5ex]{\linewidth}{1pt} }
0.1µW to 20W, 2 dB increments PRN: L1, G1
\subsubsection*{Power or power range}
Min: 1e-07W\\Max: 20W\subsubsection*{Test bands/constellation}
'L1', 'G1'
\subsubsection*{Transmitter equpment}
'F8.1'
\\\subsection{1.6.3  0.1µW to 20W, 2 dB increments PRN: L1, G1, L2}

\textcolor{lightgray}{\noindent\rule[0.5ex]{\linewidth}{1pt} }
0.1µW to 20W, 2 dB increments PRN: L1, G1, L2
\subsubsection*{Power or power range}
Min: 1e-07W\\Max: 20W\subsubsection*{Test bands/constellation}
'L1', 'G1', 'L2'
\subsubsection*{Transmitter equpment}
'F8.1'
\\\subsection{1.6.4  0.1µW to 20W, 2 dB increments PRN: L1, G1, L2, L5}

\textcolor{lightgray}{\noindent\rule[0.5ex]{\linewidth}{1pt} }
0.1µW to 20W, 2 dB increments PRN: L1, G1, L2, L5
\subsubsection*{Power or power range}
Min: 1e-07W\\Max: 20W\subsubsection*{Test bands/constellation}
'L1', 'G1', 'L2', 'L5'
\subsubsection*{Transmitter equpment}
'F8.1'
\\\section{1.7: Stationary high-power jamming, ramp power with CW}

\subsection*{Rationale}

The main objective is to observe how the J/S signal affect the loss of PNT, and/or how it produces inaccurate PNT data, and at which power level. This will allow for evaluation of the sensitivity thresholds for various systems.

\subsection*{Test description}

The attendees should be at a stationary location with a known distance to the jammer, so they can observe how different levels will affect the PNT.

\subsection*{Additional information}

The jammer employed will be F8.1 "Porcus Major", see appendix A.

\section*{Test within this testgroup}

\subsection{1.7.1  0.1µW to 20W, 2 dB increments CW: L1}

\textcolor{lightgray}{\noindent\rule[0.5ex]{\linewidth}{1pt} }
0.1µW to 20W, 2 dB increments CW: L1
\subsubsection*{Power or power range}
Min: 1e-07W\\Max: 20W\subsubsection*{Test bands/constellation}
'L1'
\subsubsection*{Transmitter equpment}
'F8.1'
\\\subsection{1.7.2  0.1µW to 20W, 2 dB increments CW: L1, G1}

\textcolor{lightgray}{\noindent\rule[0.5ex]{\linewidth}{1pt} }
0.1µW to 20W, 2 dB increments CW: L1, G1
\subsubsection*{Power or power range}
Min: 1e-07W\\Max: 20W\subsubsection*{Test bands/constellation}
'L1', 'G1'
\subsubsection*{Transmitter equpment}
'F8.1'
\\\subsection{1.7.3  0.1µW to 20W, 2 dB increments CW: L1, G1, L2}

\textcolor{lightgray}{\noindent\rule[0.5ex]{\linewidth}{1pt} }
0.1µW to 20W, 2 dB increments CW: L1, G1, L2
\subsubsection*{Power or power range}
Min: 1e-07W\\Max: 20W\subsubsection*{Test bands/constellation}
'L1', 'G1', 'L2'
\subsubsection*{Transmitter equpment}
'F8.1'
\\\subsection{1.7.4  0.1µW to 20W, 2 dB increments CW: L1, G1, L2, L5}

\textcolor{lightgray}{\noindent\rule[0.5ex]{\linewidth}{1pt} }
0.1µW to 20W, 2 dB increments CW: L1, G1, L2, L5
\subsubsection*{Power or power range}
Min: 1e-07W\\Max: 20W\subsubsection*{Test bands/constellation}
'L1', 'G1', 'L2', 'L5'
\subsubsection*{Transmitter equpment}
'F8.1'
\\\section{1.8: Stationary pyramid jamming with PRN for all GNSS bands sequentially}

\subsection*{Rationale}

This ‘pyramid’ is intended to test the potential fallback behaviour of modern multi-constellation multi frequency receivers.

\subsection*{Test description}

The jamming is performed with PRN modulation. The tests will jam most GNSS bands, incrementally adding bands to the list of jammed signals, then removing them in the reverse order. After the last test, a break should be added to allow receivers to default back to normal.

\subsection*{Additional information}

The jammer employed will be F8.1 "Porcus Major", see appendix A.

\section*{Test within this testgroup}

\subsection{1.8.1  20W PRN: E6}

\textcolor{lightgray}{\noindent\rule[0.5ex]{\linewidth}{1pt} }
20W PRN: E6
\subsubsection*{Power or power range}
Min: 20W\\Max: 20W\subsubsection*{Test bands/constellation}
'E6'
\subsubsection*{Transmitter equpment}
'F8.1'
\\\subsection{1.8.2  20W PRN: E6, E5b}

\textcolor{lightgray}{\noindent\rule[0.5ex]{\linewidth}{1pt} }
20W PRN: E6, E5b
\subsubsection*{Power or power range}
Min: 20W\\Max: 20W\subsubsection*{Test bands/constellation}
'E6', 'E5b'
\subsubsection*{Transmitter equpment}
'F8.1'
\\\subsection{1.8.3  20W PRN: E6, E5b, L5}

\textcolor{lightgray}{\noindent\rule[0.5ex]{\linewidth}{1pt} }
20W PRN: E6, E5b, L5
\subsubsection*{Power or power range}
Min: 20W\\Max: 20W\subsubsection*{Test bands/constellation}
'E6', 'E5b', 'L5'
\subsubsection*{Transmitter equpment}
'F8.1'
\\\subsection{1.8.4  20W PRN: E6, E5b, L5, G2}

\textcolor{lightgray}{\noindent\rule[0.5ex]{\linewidth}{1pt} }
20W PRN: E6, E5b, L5, G2
\subsubsection*{Power or power range}
Min: 20W\\Max: 20W\subsubsection*{Test bands/constellation}
'E6', 'E5b', 'L5', 'G2'
\subsubsection*{Transmitter equpment}
'F8.1'
\\\subsection{1.8.5  20W PRN: E6, E5b, L5, G2, L2}

\textcolor{lightgray}{\noindent\rule[0.5ex]{\linewidth}{1pt} }
20W PRN: E6, E5b, L5, G2, L2
\subsubsection*{Power or power range}
Min: 20W\\Max: 20W\subsubsection*{Test bands/constellation}
'E6', 'E5b', 'L5', 'G2', 'L2'
\subsubsection*{Transmitter equpment}
'F8.1'
\\\subsection{1.8.6  20W PRN: E6, E5b, L5, G2, L2, B1l}

\textcolor{lightgray}{\noindent\rule[0.5ex]{\linewidth}{1pt} }
20W PRN: E6, E5b, L5, G2, L2, B1l
\subsubsection*{Power or power range}
Min: 20W\\Max: 20W\subsubsection*{Test bands/constellation}
'E6', 'E5b', 'L5', 'G2', 'L2', 'B1l'
\subsubsection*{Transmitter equpment}
'F8.1'
\\\subsection{1.8.7  20W PRN: E6, E5b, L5, G2, L2, B1l, G1}

\textcolor{lightgray}{\noindent\rule[0.5ex]{\linewidth}{1pt} }
20W PRN: E6, E5b, L5, G2, L2, B1l, G1
\subsubsection*{Power or power range}
Min: 20W\\Max: 20W\subsubsection*{Test bands/constellation}
'E6', 'E5b', 'L5', 'G2', 'L2', 'B1l', 'G1'
\subsubsection*{Transmitter equpment}
'F8.1'
\\\subsection{1.8.8  20W PRN: E6, E5b, L5, G2, L2, B1l, G1, L1}

\textcolor{lightgray}{\noindent\rule[0.5ex]{\linewidth}{1pt} }
20W PRN: E6, E5b, L5, G2, L2, B1l, G1, L1
\subsubsection*{Power or power range}
Min: 20W\\Max: 20W\subsubsection*{Test bands/constellation}
'E6', 'E5b', 'L5', 'G2', 'L2', 'B1l', 'G1', 'L1'
\subsubsection*{Transmitter equpment}
'F8.1'
\\\section{1.9: Stationary inverted pyramid jamming with PRN for all GNSS bands sequentially}

\subsection*{Rationale}

This ‘inverted pyramid’ is intended to test the potential fallback behaviour of modern multi-constellation multi frequency receivers.

\subsection*{Test description}

The jamming is performed with PRN modulation. The tests will jam most GNSS bands, incrementally removing bands to the list of jammed signals, then adding them in the reverse order. The test will start with removing L1 and continue removing bands according to the list in the test name until only E5b remains. The test will continue by adding the bands back in the reverse order ending with L1. After the last test, a break should be added to allow receivers to default back to normal.

\subsection*{Additional information}

The jammer employed will be F8.1 "Porcus Major", see appendix A.

\section*{Test within this testgroup}

\subsection{1.9.1  20W PRN: E5b, L5, E6, G2, L2, B1l, G1, L1}

\textcolor{lightgray}{\noindent\rule[0.5ex]{\linewidth}{1pt} }
20W PRN: E5b, L5, E6, G2, L2, B1l, G1, L1 The test will start with removing L1 and continue removing bands according to the list in the test name until only E5b remains. The test will continue by adding the bands back in the reverse order ending with L1.
\subsubsection*{Power or power range}
Min: 20W\\Max: 20W\subsubsection*{Test bands/constellation}
'E6', 'E5b', 'L5', 'G2', 'L2', 'B1l', 'G1', 'L1'
\subsubsection*{Transmitter equpment}
'F8.1'
\\\section{1.10: Motorcade with low-power commercially available jammers (placed on stationary vehicle)}

\subsection*{Rationale}

These tests will explore the impact on other cars caused by a jammer placed in a parked car.

\subsection*{Test description}

Jammers used in this test are commercially available jammers. The jammers are to be placed on the roof of a vehicle.

\section*{Test within this testgroup}

\subsection{1.10.1  Driving while passing a parked car with GPS (L1 \& L2) jammer - jammer S2.1}

\textcolor{lightgray}{\noindent\rule[0.5ex]{\linewidth}{1pt} }
Test with jammer S2.1
\subsubsection*{Power or power range}
Min: 0.0316W\\Max: 0.1W\subsubsection*{Test bands/constellation}
'L1', ' E1', ' B1l', ' B1C', ' L5', ' E5a/b', ' B2a/b', ' G3'
\subsubsection*{Transmitter equpment}
'S2.1'
\\\subsection{1.10.2  Driving while passing a parked car with multi-band jammer - jammer H6.4}

\textcolor{lightgray}{\noindent\rule[0.5ex]{\linewidth}{1pt} }
Jammer H6.4
\subsubsection*{Power or power range}
Min: 1W\\Max: 1.58W\subsubsection*{Test bands/constellation}
'L5', ' B2a', ' E5a', ' L2', ' G2', ' B3l', ' E6', ' L1', ' E1', ' B1C', ' B1l', ' G1'
\subsubsection*{Transmitter equpment}
'H6.4'
\\\subsection{1.10.3  Vehicle starting in GPS (L1 \& L2) denied environment - jammer S2.1}

\textcolor{lightgray}{\noindent\rule[0.5ex]{\linewidth}{1pt} }
Test with jammer S2.1
\subsubsection*{Power or power range}
Min: 0.0316W\\Max: 0.1W\subsubsection*{Test bands/constellation}
'L1', ' E1', ' B1l', ' B1C', ' L5', ' E5a/b', ' B2a/b', ' G3'
\subsubsection*{Transmitter equpment}
'S2.1'
\\\subsection{1.10.4  Vehicle starting in multi-band denied environment - jammer H6.4}

\textcolor{lightgray}{\noindent\rule[0.5ex]{\linewidth}{1pt} }
Jammer H6.4
\subsubsection*{Power or power range}
Min: 1W\\Max: 1.58W\subsubsection*{Test bands/constellation}
'L5', ' B2a', ' E5a', ' L2', ' G2', ' B3l', ' E6', ' L1', ' E1', ' B1C', ' B1l', ' G1'
\subsubsection*{Transmitter equpment}
'H6.4'
\\\section{1.11: Motorcade with low-power commercially available jammers (mobile placement in cars)}

\subsection*{Rationale}

This setup is to simulate meeting a vehicle with a jammer inside of it.

\subsection*{Test description}

Jammers used in this test are commercially available jammers.

\section*{Test within this testgroup}

\subsection{1.11.1  Driving with GPS (L1 \& L2) jammer in test vehicle - jammer S2.1}

\textcolor{lightgray}{\noindent\rule[0.5ex]{\linewidth}{1pt} }
Test with jammer S2.1
\subsubsection*{Power or power range}
Min: 0.0316W\\Max: 0.1W\subsubsection*{Test bands/constellation}
'L1', ' E1', ' B1l', ' B1C', ' L5', ' E5a/b', ' B2a/b', ' G3'
\subsubsection*{Transmitter equpment}
'S2.1'
\\\subsection{1.11.2  Driving with GPS (L1 \& L2) jammer in vehicle in front of the test vehicle - jammer S2.1}

\textcolor{lightgray}{\noindent\rule[0.5ex]{\linewidth}{1pt} }
Test with jammer S2.1
\subsubsection*{Power or power range}
Min: 0.0316W\\Max: 0.1W\subsubsection*{Test bands/constellation}
'L1', ' E1', ' B1l', ' B1C', ' L5', ' E5a/b', ' B2a/b', ' G3'
\subsubsection*{Transmitter equpment}
'S2.1'
\\\subsection{1.11.3  Driving with GPS (L1 \& L2) jammer in vehicle behind the test vehicle - jammer S2.1}

\textcolor{lightgray}{\noindent\rule[0.5ex]{\linewidth}{1pt} }
Test with jammer S2.1
\subsubsection*{Power or power range}
Min: 0.0316W\\Max: 0.1W\subsubsection*{Test bands/constellation}
'L1', ' E1', ' B1l', ' B1C', ' L5', ' E5a/b', ' B2a/b', ' G3'
\subsubsection*{Transmitter equpment}
'S2.1'
\\\subsection{1.11.4  Driving with GPS (L1 \& L2) jammer in vehicle overtaking the test vehicle - jammer S2.1}

\textcolor{lightgray}{\noindent\rule[0.5ex]{\linewidth}{1pt} }
Test with jammer S2.1
\subsubsection*{Power or power range}
Min: 0.0316W\\Max: 0.1W\subsubsection*{Test bands/constellation}
'L1', ' E1', ' B1l', ' B1C', ' L5', ' E5a/b', ' B2a/b', ' G3'
\subsubsection*{Transmitter equpment}
'S2.1'
\\\subsection{1.11.5  Driving with GPS (L1 \& L2) jammer in vehicle being overtaken by the test vehicle jammer S2.1}

\textcolor{lightgray}{\noindent\rule[0.5ex]{\linewidth}{1pt} }
Test with jammer S2.1
\subsubsection*{Power or power range}
Min: 0.0316W\\Max: 0.1W\subsubsection*{Test bands/constellation}
'L1', ' E1', ' B1l', ' B1C', ' L5', ' E5a/b', ' B2a/b', ' G3'
\subsubsection*{Transmitter equpment}
'S2.1'
\\\subsection{1.11.6  Driving with multi-band jammer in test vehicle - jammer H6.4}

\textcolor{lightgray}{\noindent\rule[0.5ex]{\linewidth}{1pt} }
Jammer H6.4
\subsubsection*{Power or power range}
Min: 1W\\Max: 1.58W\subsubsection*{Test bands/constellation}
'L5', ' B2a', ' E5a', ' L2', ' G2', ' B3l', ' E6', ' L1', ' E1', ' B1C', ' B1l', ' G1'
\subsubsection*{Transmitter equpment}
'H6.4'
\\\subsection{1.11.7  Driving with multi-band jammer in vehicle in front of the test vehicle - jammer H6.4}

\textcolor{lightgray}{\noindent\rule[0.5ex]{\linewidth}{1pt} }
Jammer H6.4
\subsubsection*{Power or power range}
Min: 1W\\Max: 1.58W\subsubsection*{Test bands/constellation}
'L5', ' B2a', ' E5a', ' L2', ' G2', ' B3l', ' E6', ' L1', ' E1', ' B1C', ' B1l', ' G1'
\subsubsection*{Transmitter equpment}
'H6.4'
\\\subsection{1.11.8  Driving with multi-band jammer in vehicle behind the test vehicle - jammer H6.4}

\textcolor{lightgray}{\noindent\rule[0.5ex]{\linewidth}{1pt} }
Jammer H6.4
\subsubsection*{Power or power range}
Min: 1W\\Max: 1.58W\subsubsection*{Test bands/constellation}
'L5', ' B2a', ' E5a', ' L2', ' G2', ' B3l', ' E6', ' L1', ' E1', ' B1C', ' B1l', ' G1'
\subsubsection*{Transmitter equpment}
'H6.4'
\\\subsection{1.11.9  Driving with multi-band jammer in vehicle overtaking the test vehicle - jammer H6.4}

\textcolor{lightgray}{\noindent\rule[0.5ex]{\linewidth}{1pt} }
Jammer H6.4
\subsubsection*{Power or power range}
Min: 1W\\Max: 1.58W\subsubsection*{Test bands/constellation}
'L5', ' B2a', ' E5a', ' L2', ' G2', ' B3l', ' E6', ' L1', ' E1', ' B1C', ' B1l', ' G1'
\subsubsection*{Transmitter equpment}
'H6.4'
\\\subsection{1.11.10  Driving with multi-band jammer in vehicle being overtaken by the test vehicle -jammer H6.4}

\textcolor{lightgray}{\noindent\rule[0.5ex]{\linewidth}{1pt} }
Jammer H6.4
\subsubsection*{Power or power range}
Min: 1W\\Max: 1.58W\subsubsection*{Test bands/constellation}
'L5', ' B2a', ' E5a', ' L2', ' G2', ' B3l', ' E6', ' L1', ' E1', ' B1C', ' B1l', ' G1'
\subsubsection*{Transmitter equpment}
'H6.4'
\\\section{1.12: Low power jamming with commercially available multi-band jammers in different placements in the terrain}

\subsection*{Rationale}

The main objective is to simulate meeting several "more dangerous" jammers, multi-band jammers.

\subsection*{Test description}

The test will use three multiband jammers, spaced out in the terrain in different places. Attendees can move around or station themselves so that they can experience the different constellation and observe how their equipment and systems behave in a complicated GNSS RFI environment.

\subsection*{Additional information}

The precise positions for each jammer will have to be decided in field, to best accommodate participants wishes and practical concerns (like terrain). The coordinates for each position, X, Y and Z, will have to be written down in field to help later analysis of the test results.

\section*{Test within this testgroup}

\subsection{1.12.1  All jammers stationary; activate Jammer F6.1, H6.5 and H3.3 sequentially}

\textcolor{lightgray}{\noindent\rule[0.5ex]{\linewidth}{1pt} }
Sequential activation of jammers. Max/min power does not account for multiple jammers being active at once.
\subsubsection*{Power or power range}
Min: 0.5012W\\Max: 6.31W\subsubsection*{Test bands/constellation}
'L5', ' B2a', ' E5a', ' L2', ' G2', ' B3l', ' E6', ' L1', ' E1', ' B1C', ' B1l', ' G1'
\subsubsection*{Transmitter equpment}
'F6.1', 'H6.5', 'H3.3'
\\\subsection{1.12.2  All jammers stationary in new placements; activate Jammer F6.1, H6.5 and H3.3 sequentially}

\textcolor{lightgray}{\noindent\rule[0.5ex]{\linewidth}{1pt} }
Sequential activation of jammers. Max/min power does not account for multiple jammers being active at once.
\subsubsection*{Power or power range}
Min: 0.5012W\\Max: 6.31W\subsubsection*{Test bands/constellation}
'L5', ' B2a', ' E5a', ' L2', ' G2', ' B3l', ' E6', ' L1', ' E1', ' B1C', ' B1l', ' G1'
\subsubsection*{Transmitter equpment}
'F6.1', 'H6.5', 'H3.3'
\\\subsection{1.12.3  Jammers F6.1 and H6.5 stationary, Jammer H3.3 mobile; all jammers activated simultaneously}

\textcolor{lightgray}{\noindent\rule[0.5ex]{\linewidth}{1pt} }
Max/min power does not account for multiple jammers being active at once.
\subsubsection*{Power or power range}
Min: 0.5012W\\Max: 6.31W\subsubsection*{Test bands/constellation}
'L5', ' B2a', ' E5a', ' L2', ' G2', ' B3l', ' E6', ' L1', ' E1', ' B1C', ' B1l', ' G1'
\subsubsection*{Transmitter equpment}
'F6.1', 'H6.5', 'H3.3'
\\\section{1.13: Jamming attacks on ships}

\subsection*{Rationale}

The objective is to simulate the conditions of which a jammer can appear on ships like ferries.

\subsection*{Test description}

Exact locations and tests will be chosen on site according to layout of ship and available time schedule.

\section*{Test within this testgroup}

\subsection{1.13.1  Mobile jammer (H8.1) (L1 only) on the car deck outside car}

\textcolor{lightgray}{\noindent\rule[0.5ex]{\linewidth}{1pt} }
Jammer H8.1
\subsubsection*{Power or power range}
Min: 0.631W\\Max: 0.631W\subsubsection*{Test bands/constellation}
'L1'
\subsubsection*{Transmitter equpment}
'H8.1'
\\\subsection{1.13.2  Mobile jammer (H8.1) (L1 only) on the car deck inside car}

\textcolor{lightgray}{\noindent\rule[0.5ex]{\linewidth}{1pt} }
Jammer H8.1
\subsubsection*{Power or power range}
Min: 0.631W\\Max: 0.631W\subsubsection*{Test bands/constellation}
'L1'
\subsubsection*{Transmitter equpment}
'H8.1'
\\\subsection{1.13.3  Mobile jammer (H6.6) (L1+L2) - on the car deck outside car}

\textcolor{lightgray}{\noindent\rule[0.5ex]{\linewidth}{1pt} }
Jammer H6.6
\subsubsection*{Power or power range}
Min: 1W\\Max: 1.58W\subsubsection*{Test bands/constellation}
'L1', ' L2'
\subsubsection*{Transmitter equpment}
'H6.6'
\\\subsection{1.13.4  Mobile jammer (H6.6) (L1+L2) - on the car deck inside car}

\textcolor{lightgray}{\noindent\rule[0.5ex]{\linewidth}{1pt} }
Jammer H6.6
\subsubsection*{Power or power range}
Min: 1W\\Max: 1.58W\subsubsection*{Test bands/constellation}
'L1', ' L2'
\subsubsection*{Transmitter equpment}
'H6.6'
\\\subsection{1.13.5  Mobile jammer (H6.6) (multi-band) - on the car deck outside car}

\textcolor{lightgray}{\noindent\rule[0.5ex]{\linewidth}{1pt} }
Jammer H6.6
\subsubsection*{Power or power range}
Min: 1W\\Max: 1.58W\subsubsection*{Test bands/constellation}
'L5', ' B2a', ' E5a', ' L2', ' G2', ' B3l', ' E6', ' L1', ' E1', ' B1C', ' B1l', ' G1'
\subsubsection*{Transmitter equpment}
'H6.6'
\\\subsection{1.13.6  Mobile jammer (H6.6) (multi-band) - on the car deck inside car}

\textcolor{lightgray}{\noindent\rule[0.5ex]{\linewidth}{1pt} }
Jammer H6.6
\subsubsection*{Power or power range}
Min: 1W\\Max: 1.58W\subsubsection*{Test bands/constellation}
'L5', ' B2a', ' E5a', ' L2', ' G2', ' B3l', ' E6', ' L1', ' E1', ' B1C', ' B1l', ' G1'
\subsubsection*{Transmitter equpment}
'H6.6'
\\\subsection{1.13.7  Mobile jammer (H6.6) (multi-band) - on deck close to the ship’s antennas (by the bridge)}

\textcolor{lightgray}{\noindent\rule[0.5ex]{\linewidth}{1pt} }
Jammer H6.6
\subsubsection*{Power or power range}
Min: 1W\\Max: 1.58W\subsubsection*{Test bands/constellation}
'L5', ' B2a', ' E5a', ' L2', ' G2', ' B3l', ' E6', ' L1', ' E1', ' B1C', ' B1l', ' G1'
\subsubsection*{Transmitter equpment}
'H6.6'
\\\subsection{1.13.8  Mobile jammer (H6.6) (multi-band) - inside public areas of boat (under the bridge)}

\textcolor{lightgray}{\noindent\rule[0.5ex]{\linewidth}{1pt} }
Jammer H6.6
\subsubsection*{Power or power range}
Min: 1W\\Max: 1.58W\subsubsection*{Test bands/constellation}
'L5', ' B2a', ' E5a', ' L2', ' G2', ' B3l', ' E6', ' L1', ' E1', ' B1C', ' B1l', ' G1'
\subsubsection*{Transmitter equpment}
'H6.6'
\\\section{1.14: Stationary high-power jamming, ramp power with PRN - Ramnan (200 W)}

\subsection*{Rationale}

The main objective is to observe how the J/S signal affect the loss of PNT, and/or how it produces inaccurate PNT data, and at which power level. This will allow for evaluation of the sensitivity thresholds for various systems.

\subsection*{Test description}

The jammer will be placed at a mountainside. This will allow for attendees to evaluate the difference between signals arriving from in the horizontal plane and signals arriving with some elevation above the horizontal. Each test will last for 15.67 minutes, with a 15-minute break between each test.

\subsection*{Additional information}

The jammer employed will be "Porcus Major" F8.1, see appendix A. The last step, from 52 dBm to 53.0103 dBm (200 W), will be a 1.0103 dB increment, not a 2 dB increment.

\section*{Test within this testgroup}

\subsection{1.14.1  0.1µW to 200 W, 2 dB increments PRN: L1}

\textcolor{lightgray}{\noindent\rule[0.5ex]{\linewidth}{1pt} }
Power ramp using F8.1
\subsubsection*{Power or power range}
Min: 1e-07W\\Max: 200W\subsubsection*{Test bands/constellation}
'L1'
\subsubsection*{Transmitter equpment}
'F8.1'
\\\subsection{1.14.2  0.1µW to 200 W, 2 dB increments PRN: L1, G1}

\textcolor{lightgray}{\noindent\rule[0.5ex]{\linewidth}{1pt} }
Power ramp using F8.1
\subsubsection*{Power or power range}
Min: 1e-07W\\Max: 200W\subsubsection*{Test bands/constellation}
'L1', 'G1'
\subsubsection*{Transmitter equpment}
'F8.1'
\\\subsection{1.14.3  0.1µW to 200 W, 2 dB increments PRN: L1, G1, L2}

\textcolor{lightgray}{\noindent\rule[0.5ex]{\linewidth}{1pt} }
Power ramp using F8.1
\subsubsection*{Power or power range}
Min: 1e-07W\\Max: 200W\subsubsection*{Test bands/constellation}
'L1', 'G1', 'L2'
\subsubsection*{Transmitter equpment}
'F8.1'
\\\subsection{1.14.4  0.1µW to 200 W, 2 dB increments PRN: L1, G1, L2, L5}

\textcolor{lightgray}{\noindent\rule[0.5ex]{\linewidth}{1pt} }
Power ramp using F8.1
\subsubsection*{Power or power range}
Min: 1e-07W\\Max: 200W\subsubsection*{Test bands/constellation}
'L1', 'G1', 'L2', 'L5'
\subsubsection*{Transmitter equpment}
'F8.1'
\\\section{1.15: Stationary low-power jamming of L1-only and G1-only}

\subsection*{Rationale}

The main objective is to test receivers’ ability to change between using GPS and Glonass when one or the other is denied.

\subsection*{Test description}

A 20 MHz wideband (WB) white noise signal will be active on either L1 or G1. Signal power will be ramped up during the first test, and then kept at the achieved maximum power for the reminder of the tests.

\subsection*{Additional information}

Each test will have a short break after it is completed. When L1-only and G1-only is combined in a test, the transmission will change from the first to the second instantly.

\section*{Test within this testgroup}

\subsection{1.15.1  WB, L1-only}

\textcolor{lightgray}{\noindent\rule[0.5ex]{\linewidth}{1pt} }
Low-power jamming
\subsubsection*{Power or power range}
Min: 0.1W\\Max: 1W\subsubsection*{Test bands/constellation}
'L1'
\subsubsection*{Transmitter equpment}
'N/A'
\\\subsection{1.15.2  WB, G1-only}

\textcolor{lightgray}{\noindent\rule[0.5ex]{\linewidth}{1pt} }
Low-power jamming
\subsubsection*{Power or power range}
Min: 0.1W\\Max: 1W\subsubsection*{Test bands/constellation}
'G1'
\subsubsection*{Transmitter equpment}
'N/A'
\\\subsection{1.15.3  WB, G1-only then L1-only}

\textcolor{lightgray}{\noindent\rule[0.5ex]{\linewidth}{1pt} }
Low-power jamming
\subsubsection*{Power or power range}
Min: 0.1W\\Max: 1W\subsubsection*{Test bands/constellation}
'G1', 'L1'
\subsubsection*{Transmitter equpment}
'N/A'
\\\subsection{1.15.4  WB, L1-only then G1-only}

\textcolor{lightgray}{\noindent\rule[0.5ex]{\linewidth}{1pt} }
Low-power jamming
\subsubsection*{Power or power range}
Min: 0.1W\\Max: 1W\subsubsection*{Test bands/constellation}
'G1', 'L1'
\subsubsection*{Transmitter equpment}
'N/A'
\\\chapter{2 Spoofing}

\section{2.1: Incoherent spoofing from stationary spoofer using synthetic ephemerides}

\subsection*{Rationale}

These are very basic attacks that can be performed with easily available software and hardware. These attacks can give an indication to the receivers’ resiliency to spoofing attacks. Most receivers will probably see these attacks as noise initially, effectively working as a jamming signal.

\subsection*{Test description}

Simulated signals will be transmitted from a stationary antenna. Generated spoofing scenarios will use satellite ephemerides different from live sky satellites. Simulated signals may use one or more constellations and one or more signal bands.

Initial positions are either False (e.g. 70 N, 10 E) or True (target location at transmitter antenna location). Initial time is either False (e.g. a jump in time) or True ( < 100 ns timing error for a receiver at target location). Some test scenarios may be started with jamming (lasting for 5 min, one or several signal bands, before the spoofing transmission is activated). Some spoofing scenarios may be accompanied 
by continuous jamming (one or several signal bands).
Static scenarios are a fixed position, while motion scenarios are a drive around the area. For each dynamic test, the motion is first spoofed to a fixed start position for 5 minutes before the dynamic motion starts.

\subsection*{Additional information}

Expected range/power of spoofing signals: A radius of approximately 1.5 kilometre from the transmitter, depending on terrain and building signal shielding.

\section*{Test within this testgroup}

\subsection{2.1.1  Large position and time jump, gradually increasing signal strength}

\textcolor{lightgray}{\noindent\rule[0.5ex]{\linewidth}{1pt} }
Signals: 
GPS L1 C/A, L2C, L5 
Galileo E1, E5 
No jamming 
Simulated position: 70 N, 10 E 
Simulated start time: 01.10.2023 12:00
\subsubsection*{Power or power range}
Min: 1W\\Max: 100W\subsubsection*{Test bands/constellation}
'L1', 'L2', 'L5', 'E1', 'E5a', 'E5b'
\subsubsection*{Transmitter equpment}
'N/A'
\\\subsection{2.1.2  Large position and time jump}

\textcolor{lightgray}{\noindent\rule[0.5ex]{\linewidth}{1pt} }
Signals: 
GPS L1 C/A 
Galileo E1 
No jamming 
Position: 70 N, 10 E 
Simulated start time: 01.10.2023 12:00
\subsubsection*{Power or power range}
Min: 1W\\Max: 100W\subsubsection*{Test bands/constellation}
'L1', 'E1'
\subsubsection*{Transmitter equpment}
'N/A'
\\\subsection{2.1.3  Large position and time jump, with jamming}

\textcolor{lightgray}{\noindent\rule[0.5ex]{\linewidth}{1pt} }
Signals: 
GPS L1 C/A 
Galileo E1 
5 minutes of initial jamming (L1, G1, B1l, E6, L2, E5b, L5 with 2 W) prior to spoofing transmission, then continuous on other bands than the ones spoofed. 
Simulated position: 70 N, 10 E 
Simulated start time: 01.10.2023 12:00
\subsubsection*{Power or power range}
Min: 1W\\Max: 100W\subsubsection*{Test bands/constellation}
'L1', 'E1'
\subsubsection*{Transmitter equpment}
'N/A'
\\\subsection{2.1.4  Simulated driving (route 1)}

\textcolor{lightgray}{\noindent\rule[0.5ex]{\linewidth}{1pt} }
Signals: GPS L1 C/A, L2C, L5 
Galileo E1, E5 
5 minutes of initial jamming (L1, G1, B1l, E6, L2, E5b, L5 with 2 W) prior to spoofing transmission. 
Simulated start position: Transmitter location 
Simulated start time: 01.10.2023 12:00
\subsubsection*{Power or power range}
Min: 1W\\Max: 100W\subsubsection*{Test bands/constellation}
'L1', 'L2', 'L5', 'E1', 'E5a', 'E5b'
\subsubsection*{Transmitter equpment}
'N/A'
\\\subsection{2.1.5  Simulated driving, true reference time (route 1)}

\textcolor{lightgray}{\noindent\rule[0.5ex]{\linewidth}{1pt} }
Signals: GPS L1 C/A, L2C, L5 
Galileo E1, E5 
5 minutes of initial jamming (L1, G1, B1l, E6, L2, E5b, L5 with 2 W) prior to spoofing transmission. 
Simulated start position: Transmitter location 
Simulated start time: Referenced to live GPS-signals
\subsubsection*{Power or power range}
Min: 1W\\Max: 100W\subsubsection*{Test bands/constellation}
'L1', 'L2', 'L5', 'E1', 'E5a', 'E5b'
\subsubsection*{Transmitter equpment}
'N/A'
\\\section{2.2: Incoherent spoofing from stationary spoofer using broadcast(true) ephemerides}

\subsection*{Rationale}

These spoofing tests use ephemerides (navigation data) identical to those broadcasted by the actual satellites, but the transmitted spoofing signals do not align with those received from actual satellites. Receivers using the spoofed signals will generate jumps in the navigation solution, either in position, timing and/or velocity.

\subsection*{Test description}

Simulated signals will be transmitted from a stationary antenna. Generated spoofing scenarios will use broadcast satellite ephemeris data. Simulated signals may use 
one or more constellations and one or more signal bands.

Initial positions are either False (e.g. 70 N, 10 E) or True (target location at transmitter antenna location). Initial time is either False (e.g. a jump in time/date) or True ( < 100 ns timing error for a receiver at target location). Some test scenarios may be started with jamming (lasting for 5 min, one or several signal bands, before the spoofing transmission is activated). Some spoofing scenarios may be 
accompanied by continuous jamming (one or several signal bands). 

Static scenarios are a fixed position, while motion scenarios are a simulated drive around the area. There will be a break between each test to allow receivers to reacquire fix onto real satellite signals. For each dynamic test, the motion is first spoofed to a fixed start position for 5 minutes before the dynamic motion starts.

\subsection*{Additional information}

Expected range/power of spoofing signals: A radius of approximately 1.5 kilometre from the transmitter, depending on terrain and building signal shielding.

\section*{Test within this testgroup}

\subsection{2.2.1  Large position jump}

\textcolor{lightgray}{\noindent\rule[0.5ex]{\linewidth}{1pt} }
Signals: 
GPS L1 C/A, L2C, L5 
Galileo E1, E5 
No jamming 
Simulated position: 70 N, 10 E 
Simulated start time: Referenced to live GPS-signals
\subsubsection*{Power or power range}
Min: 1W\\Max: 100W\subsubsection*{Test bands/constellation}
'L1', 'L2', 'L5', 'E1', 'E5a', 'E5b'
\subsubsection*{Transmitter equpment}
'N/A'
\\\subsection{2.2.2  Small position jump, large time jump}

\textcolor{lightgray}{\noindent\rule[0.5ex]{\linewidth}{1pt} }
Signals: 
GPS L1 C/A, L2C, L5 
Galileo E1, E5 
5 minutes of initial jamming (L1, G1, B1l, E6, L2, E5b, L5 with 2 W) prior to spoofing transmission, then continuous on other bands than the ones spoofed. 
Simulated position: North end of the football field - 69.27701401, 15.969328354, 45 m hae. (Height Above Ellipsoid) 
Simulated start time: 01.10.2023 12:00
\subsubsection*{Power or power range}
Min: 1W\\Max: 100W\subsubsection*{Test bands/constellation}
'L1', 'L2', 'L5', 'E1', 'E5a', 'E5b'
\subsubsection*{Transmitter equpment}
'N/A'
\\\subsection{2.2.3  Small position jump}

\textcolor{lightgray}{\noindent\rule[0.5ex]{\linewidth}{1pt} }
Signals: 
GPS L1 C/A, L2C, L5 
Galileo E1, E5 
No jamming 
Simulated position: North end of the football field - 69.27701401, 15.96932835, 45 m hae. (Height Above Ellipsoid) 
Simulated start time: Referenced to live GPS-signals
\subsubsection*{Power or power range}
Min: 1W\\Max: 100W\subsubsection*{Test bands/constellation}
'L1', 'L2', 'L5', 'E1', 'E5a', 'E5b'
\subsubsection*{Transmitter equpment}
'N/A'
\\\subsection{2.2.4  Flying (route 2) - "helicopter scenario"}

\textcolor{lightgray}{\noindent\rule[0.5ex]{\linewidth}{1pt} }
Signals: 
GPS L1 C/A, L2C, L5 
Galileo E1, E5 
No jamming 
Simulated start position: Over the sea 1 km N (Midnattskjæran) at 200 m height 
Simulated start time: Referenced to live GPS-signals 
Spoofing transmission will be corrected for signal delay to simulated start position. Helicopter at start position should see coherent signals.
\subsubsection*{Power or power range}
Min: 1W\\Max: 100W\subsubsection*{Test bands/constellation}
'L1', 'L2', 'L5', 'E1', 'E5a', 'E5b'
\subsubsection*{Transmitter equpment}
'N/A'
\\\subsection{2.2.5  Fixed position}

\textcolor{lightgray}{\noindent\rule[0.5ex]{\linewidth}{1pt} }
Signals: 
GPS L1 C/A, L2C, L5 
Galileo E1, E5 
No jamming 
Simulated position: Cemetery - 69.2824699, 15.9906568, 48 m hae. (Height Above Ellipsoid) 
Simulated start time: Referenced to live GPS-signals
\subsubsection*{Power or power range}
Min: 1W\\Max: 100W\subsubsection*{Test bands/constellation}
'L1', 'L2', 'L5', 'E1', 'E5a', 'E5b'
\subsubsection*{Transmitter equpment}
'N/A'
\\\subsection{2.2.6  Large position jump \#2}

\textcolor{lightgray}{\noindent\rule[0.5ex]{\linewidth}{1pt} }
Signals: 
GPS L1 C/A, L2C, L5 
Galileo E1, E5 
No jamming 
Simulated position: 69.25 N, 14,9 E 
Simulated start time: Referenced to live GPS-signals
\subsubsection*{Power or power range}
Min: 1W\\Max: 100W\subsubsection*{Test bands/constellation}
'L1', 'L2', 'L5', 'E1', 'E5a', 'E5b'
\subsubsection*{Transmitter equpment}
'N/A'
\\\section{2.3: Coherent spoofing from stationary spoofer using broadcast(true) ephemerides}

\subsection*{Rationale}

These spoofing tests use ephemerides (navigation data) identical to those broadcasted by the actual satellites. The transmitted spoofing signals are intended to align (to within a few 100 ns) with those received from actual satellites at the target location. Receivers using the spoofed signals at rest at the target location will initially generate no major changes in the navigation solution, either in position, timing and/or velocity, compared to the solution estimated from actual satellite signals.

\subsection*{Test description}

Simulated signals will be transmitted from a stationary antenna. Generated spoofing scenarios will use broadcast satellite ephemeris data. Simulated signals may use 
one or more constellations and one or more signal bands.

Initial positions are True (target location at transmitter antenna location). Initial time is True ( < 100 ns timing error for a receiver at target location). Some test scenarios may be started with jamming (lasting for 5 min, one or several signal bands, before the spoofing transmission is activated). Some spoofing scenarios may be 
accompanied by continuous jamming (one or several signal bands). 

For all tests in this group, spoofing transmission will be corrected for signal delay to simulated start position.

Static scenarios are a fixed position, while motion scenarios are a simulated drive around the area. There will be a break between each test to allow receivers to reacquire fix onto real satellite signals. For each dynamic test, the motion is first spoofed to a fixed start position for 5 minutes before the dynamic motion starts.

\subsection*{Additional information}

Expected range/power of spoofing signals: A radius of approximately 1.5 kilometre from the transmitter, depending on terrain and building signal shielding.

\section*{Test within this testgroup}

\subsection{2.3.1  Simulated driving (route 1). GPS only with initial jamming.}

\textcolor{lightgray}{\noindent\rule[0.5ex]{\linewidth}{1pt} }
Signals: 
GPS L1 C/A, L2C, L5 
5 minutes of initial jamming (L1, G1, B1l, E6, L2, E5b, L5 with 2 W) prior to spoofing transmission. 
Simulated start position: Bleik community house parking lot 
Simulated start time: Referenced to live GPS-signals
\subsubsection*{Power or power range}
Min: 1W\\Max: 100W\subsubsection*{Test bands/constellation}
'L1', 'L2', 'L5'
\subsubsection*{Transmitter equpment}
'N/A'
\\\subsection{2.3.2  Simulated driving (route 1). Galileo only with initial jamming.}

\textcolor{lightgray}{\noindent\rule[0.5ex]{\linewidth}{1pt} }
Signals: Galileo E1, E5 
5 minutes of initial jamming (L1, G1, B1l, E6, L2, E5b, L5 with 2 W) prior to spoofing transmission. 
Simulated start position: Bleik community house parking lot 
Simulated start time: Referenced to live GPS-signals
\subsubsection*{Power or power range}
Min: 1W\\Max: 100W\subsubsection*{Test bands/constellation}
'E1', 'E5a', 'E5b'
\subsubsection*{Transmitter equpment}
'N/A'
\\\subsection{2.3.3  Simulated driving (route 1) with initial jamming.}

\textcolor{lightgray}{\noindent\rule[0.5ex]{\linewidth}{1pt} }
Signals: 
GPS L1 C/A, L2C, L5 
Galileo E1, E5 
5 minutes of initial jamming (L1, G1, B1l, E6, L2, E5b, L5 with 2 W) prior to spoofing transmission. 
Simulated start position: Bleik community house parking lot 
Simulated start time: Referenced to live GPS-signals
\subsubsection*{Power or power range}
Min: 1W\\Max: 100W\subsubsection*{Test bands/constellation}
'L1', 'L2', 'L5', 'E1', 'E5a', 'E5b'
\subsubsection*{Transmitter equpment}
'N/A'
\\\subsection{2.3.4  Simulated driving (route 1). GPS only.}

\textcolor{lightgray}{\noindent\rule[0.5ex]{\linewidth}{1pt} }
Signals: 
GPS L1 C/A, L2C, L5 
No jamming 
Simulated start position: Bleik community house parking lot 
Simulated start time: Referenced to live GPS-signals
\subsubsection*{Power or power range}
Min: 1W\\Max: 100W\subsubsection*{Test bands/constellation}
'L1', 'L2', 'L5'
\subsubsection*{Transmitter equpment}
'N/A'
\\\subsection{2.3.5  Simulated driving (route 1). GPS L1 and Galileo E1.}

\textcolor{lightgray}{\noindent\rule[0.5ex]{\linewidth}{1pt} }
Signals: 
GPS L1 C/A 
Galileo E1 
No jamming 
Simulated start position: Bleik community house parking lot 
Simulated start time: Referenced to live GPS-signals
\subsubsection*{Power or power range}
Min: 1W\\Max: 100W\subsubsection*{Test bands/constellation}
'L1', 'E1'
\subsubsection*{Transmitter equpment}
'N/A'
\\\subsection{2.3.6  Simulated driving (route 1)}

\textcolor{lightgray}{\noindent\rule[0.5ex]{\linewidth}{1pt} }
Signals: 
GPS L1 C/A, L2C, L5 
Galileo E1, E5 
No jamming 
Simulated start position: Bleik community house parking lot 
Simulated start time: Referenced to live GPS-signals
\subsubsection*{Power or power range}
Min: 1W\\Max: 100W\subsubsection*{Test bands/constellation}
'L1', 'L2', 'L5', 'E1', 'E5a', 'E5b'
\subsubsection*{Transmitter equpment}
'N/A'
\\\subsection{2.3.7  Flying (route 4) - "drone scenario"}

\textcolor{lightgray}{\noindent\rule[0.5ex]{\linewidth}{1pt} }
Signals: 
GPS L1 C/A, L2C, L5 
Galileo E1, E5 
No jamming 
Simulated start position: 69.277014014, 15.969328354, 40 mhae. 
Simulated start time: Referenced to live GPS-signals
\subsubsection*{Power or power range}
Min: 1W\\Max: 100W\subsubsection*{Test bands/constellation}
'L1', 'L2', 'L5', 'E1', 'E5a', 'E5b'
\subsubsection*{Transmitter equpment}
'N/A'
\\\subsection{2.3.8  Sailing (route 5) - "ship scenario"}

\textcolor{lightgray}{\noindent\rule[0.5ex]{\linewidth}{1pt} }
Signals: 
GPS L1 C/A, L2C, L5 
Galileo E1, E5 
No jamming 
Simulated start position: Bleik harbour 
Simulated start time: Referenced to live GPS-signals
\subsubsection*{Power or power range}
Min: 1W\\Max: 100W\subsubsection*{Test bands/constellation}
'L1', 'L2', 'L5', 'E1', 'E5a', 'E5b'
\subsubsection*{Transmitter equpment}
'N/A'
\\\section{2.4: Incoherent time spoofing from stationary spoofer using synthetic ephemerides}

\subsection*{Rationale}

These are synchronized spoofing scenarios in the sense that the navigation solution (position, velocity and clock bias) should not initially change significantly for a receiver at the target location. The scenarios are incoherent in the sense that spoofing signals are different from those received from the actual satellites.

\subsection*{Test description}

Simulated signals will be transmitted from a stationary antenna. Generated spoofing scenarios will use satellite ephemerides different from live sky satellites. Simulated signals may use one or more constellations and one or more signal bands. 

Initial positions are True (target location at transmitter antenna location). Some test scenarios may be started with jamming (lasting for 5 min, one or several signal bands). Some spoofing scenarios may be accompanied by continuous jamming (one or several signal bands). 

There will be a small break between each test and a larger break after the test group is over to allow receivers to reacquire fix onto real satellite signals.

\subsection*{Additional information}

Expected range/power of spoofing signals: A radius of approximately few hundred metres from the transmitter, depending on terrain and building signal shielding.

\section*{Test within this testgroup}

\subsection{2.4.1  Time offset 15 minutes from real time. GPS L1 and Galileo E1}

\textcolor{lightgray}{\noindent\rule[0.5ex]{\linewidth}{1pt} }
Signals: GPS L1 C/A and Galileo E1 only.
\subsubsection*{Power or power range}
Min: 1W\\Max: 100W\subsubsection*{Test bands/constellation}
'L1', 'E1'
\subsubsection*{Transmitter equpment}
'N/A'
\\\subsection{2.4.2  Time offset 15 minutes from real time.}

\textcolor{lightgray}{\noindent\rule[0.5ex]{\linewidth}{1pt} }
Signals: 
GPS L1 C/A, L2C, L5 
Galileo E1, E5 
No jamming. 
Fixed spoofed position: 69.27547832, 15.96832496, 35 m hae. 
Time offset is + 15 minutes (900 seconds), so "into the future". 
Spoofing power ramp -35 dBm to +15 dBm in steps of 5 dB every two minutes.
\subsubsection*{Power or power range}
Min: 1W\\Max: 100W\subsubsection*{Test bands/constellation}
'L1', 'L2', 'L5', 'E1', 'E5a', 'E5b'
\subsubsection*{Transmitter equpment}
'N/A'
\\\subsection{2.4.3  Time offset -3 minutes from real time}

\textcolor{lightgray}{\noindent\rule[0.5ex]{\linewidth}{1pt} }
Signals:
GPS L1 C/A, L2C, L5 
Galileo E1, E5 
No jamming. 
Fixed spoofed position: 69.27547832, 15.96832496, 35 m hae. 
Time offset is - 3 minutes (180 seconds), so "back into the past". 
Spoofing power will start at -20 dBm and be stepped up to 15 dBm in one step.
\subsubsection*{Power or power range}
Min: 1W\\Max: 100W\subsubsection*{Test bands/constellation}
'L1', 'L2', 'L5', 'E1', 'E5a', 'E5b'
\subsubsection*{Transmitter equpment}
'N/A'
\\\subsection{2.4.4  Static + Frequency step (spoofing signal transmission rate change). GPS L1 C/A only}

\textcolor{lightgray}{\noindent\rule[0.5ex]{\linewidth}{1pt} }
Signals: GPS L1 C/A only.
\subsubsection*{Power or power range}
Min: 1W\\Max: 100W\subsubsection*{Test bands/constellation}
'L1'
\subsubsection*{Transmitter equpment}
'N/A'
\\\subsection{2.4.5  Static + Frequency step (spoofing signal transmission rate change)}

\textcolor{lightgray}{\noindent\rule[0.5ex]{\linewidth}{1pt} }
Signals: 
GPS L1 C/A 
Galileo E1 
5 minutes of initial jamming (L1, G1, B1l, L2, E5b, L5 with 2 W) prior to spoofing transmission. 
Fixed spoofed position: 69.27547832, 15.96832496, 35 m hae. 
Spoofing power will be at 0 dBm. 
Frequency steps are added (10 ns/s) and starts five minutes after the spoofing starts.
\subsubsection*{Power or power range}
Min: 1W\\Max: 100W\subsubsection*{Test bands/constellation}
'L1', 'E1'
\subsubsection*{Transmitter equpment}
'N/A'
\\\section{2.5: Coherent time spoofing from stationary spoofer using broadcast(true) ephemerides}

\subsection*{Rationale}

Scenarios in these tests is intended not to alter the navigation solution at for receivers at the target position for position and velocity estimates. Clock bias estimates should be affected by the frequency step in test 1 - 3, but not in 4 - 7.

\subsection*{Test description}

Simulated signals will be transmitted from a stationary antenna. Generated spoofing scenarios will use broadcast satellite ephemeris data. Simulated signals may use 
one or more constellations and one or more signal bands.

Initial positions are True (target location at transmitter antenna location). Initial time is True ( < 100 ns timing error for a receiver at target location). Some test scenarios may be started with jamming (lasting for 5 min, one or several signal bands). Some spoofing scenarios may be accompanied by continuous jamming (one or several signal bands). 

There will be a short break between each test and a larger break after the test group is over to allow receivers to reacquire fix onto real satellite signals.

\subsection*{Additional information}

Expected range/power of spoofing signals: A radius of approximately few hundred metres from the transmitter, depending on terrain and building signal shielding.

\section*{Test within this testgroup}

\subsection{2.5.1  Static + Frequency step (spoofing signal transmission rate change)}

\textcolor{lightgray}{\noindent\rule[0.5ex]{\linewidth}{1pt} }
Signals: 
GPS L1 C/A 
Galileo E1 
No jamming. 
Fixed spoofed position: 69.27547832, 15.96832496, 35 m hae. 
Frequency steps are added (10 ns/s), and starts five minutes after the spoofing starts. 
Spoofing power will be at -20 dBm.
\subsubsection*{Power or power range}
Min: 1W\\Max: 100W\subsubsection*{Test bands/constellation}
'L1', 'E1'
\subsubsection*{Transmitter equpment}
'N/A'
\\\subsection{2.5.2  Static + Frequency step (spoofing signal transmission rate change) with jamming}

\textcolor{lightgray}{\noindent\rule[0.5ex]{\linewidth}{1pt} }
Signals: 
GPS L1 C/A 
Galileo E1 
5 minutes of initial jamming (L1, G1, B1l, L2, E5b, L5 with 2 W) prior to spoofing transmission, then 
continuous on other bands than the ones spoofed. 
Fixed spoofed position: 69.27547832, 15.96832496, 35 m hae. 
Frequency steps are removed (10 ns/s) and starts five minutes after the spoofing starts. 
Spoofing power will be at 0 dBm.
\subsubsection*{Power or power range}
Min: 1W\\Max: 100W\subsubsection*{Test bands/constellation}
'L1', 'E1'
\subsubsection*{Transmitter equpment}
'N/A'
\\\subsection{2.5.3  Static + Nav data manipulation (clock/frequency related). L1/E1 only}

\textcolor{lightgray}{\noindent\rule[0.5ex]{\linewidth}{1pt} }
Signals: 
GPS L1 C/A 
Galileo E1 
No jamming. 
Fixed spoofed position: 69.27547832, 15.96832496, 35 m hae. 
Spoofing power will be at -20 dBm.
\subsubsection*{Power or power range}
Min: 1W\\Max: 100W\subsubsection*{Test bands/constellation}
'L1', 'E1'
\subsubsection*{Transmitter equpment}
'N/A'
\\\subsection{2.5.4  Static + Nav data manipulation (clock/frequency related). with jamming.}

\textcolor{lightgray}{\noindent\rule[0.5ex]{\linewidth}{1pt} }
Signals: 
GPS L1 C/A 
Galileo E1 
5 minutes of initial jamming (L1, G1, B1l, L2, E5b, L5 with 2 W) prior to spoofing transmission, then continuous on other bands than the ones spoofed.
Fixed spoofed position: 69.27547832, 15.96832496, 35 m hae. 
Spoofing power ramp -35 dBm to +15 dBm in steps of 5 dB every two minutes.
\subsubsection*{Power or power range}
Min: 1W\\Max: 100W\subsubsection*{Test bands/constellation}
'L1', 'E1'
\subsubsection*{Transmitter equpment}
'N/A'
\\\subsection{2.5.5  Static + UTC-parameter navigation data manipulation.}

\textcolor{lightgray}{\noindent\rule[0.5ex]{\linewidth}{1pt} }
Signals: 
GPS L1 C/A 
Galileo E1 
5 minutes of initial jamming (L1, G1, B1l, L2, E5b, L5 with 2 W) prior to spoofing transmission.
Fixed spoofed position: 69.27547832, 15.96832496, 35 m hae. 
Spoofing power will be at -20 dBm. 
Spoofing says that back in 2016, there was 19 leap seconds instead of 18.
\subsubsection*{Power or power range}
Min: 1W\\Max: 100W\subsubsection*{Test bands/constellation}
'L1', 'E1'
\subsubsection*{Transmitter equpment}
'N/A'
\\\subsection{2.5.6  Time offset 15 minutes from real time - harbour}

\textcolor{lightgray}{\noindent\rule[0.5ex]{\linewidth}{1pt} }
Signals: 
GPS L1 C/A, L2C, L5
Galileo E1, E5
No jamming. 
Fixed spoofed position: Bleik harbour 
Time offset is + 15 minutes (900 seconds), so "into the future".
\subsubsection*{Power or power range}
Min: 1W\\Max: 100W\subsubsection*{Test bands/constellation}
'L1', 'L2', 'L5', 'E1', 'E5a', 'E5b'
\subsubsection*{Transmitter equpment}
'N/A'
\\\section{2.6: Incoherent GPS position and time spoofing from mobile spoofer}

\subsection*{Rationale}

The objective is to simulate a vehicle-borne spoofing device "out in the wild", so that attendees can experience how a mobile spoofing source affects their (stationary or mobile) equipment and systems.

\subsection*{Test description}

There will be a break between each test to allow receivers to reacquire fix onto real satellite signals (total of 50 min for each test). The spoofed signals will be on GPS L1 only. All spoofing tests will be combined with jamming on Glonass G1.

\subsection*{Additional information}

Starting position will be approximately 69.194875 N, 15.837719 E in all scenarios.

\section*{Test within this testgroup}

\subsection{2.6.1  Spoofer (in vehicle) stationary with moving spoofed position.}

\textcolor{lightgray}{\noindent\rule[0.5ex]{\linewidth}{1pt} }
Spoofer (in vehicle) stationary; spoofed position starts static and approximately true. After 10 min spoofed position starts to move south with constant speed (15 m/s) while spoofer is still stationary.
\subsubsection*{Power or power range}
Min: 0.1W\\Max: 100W\subsubsection*{Test bands/constellation}
'L1'
\subsubsection*{Transmitter equpment}
'N/A'
\\\subsection{2.6.2  Spoofer (in vehicle) stationary and then moving with fixed spoofed position.}

\textcolor{lightgray}{\noindent\rule[0.5ex]{\linewidth}{1pt} }
Spoofer (in vehicle) starts stationary for 10 min, and then begins to drive south along Stavedalsveien (FV7702); spoofed position remains fixed and approximately as the true position from start throughout the test.
\subsubsection*{Power or power range}
Min: 0.1W\\Max: 100W\subsubsection*{Test bands/constellation}
'L1'
\subsubsection*{Transmitter equpment}
'N/A'
\\\subsection{2.6.3  Spoofer (in vehicle) moving with fixed spoofed position.}

\textcolor{lightgray}{\noindent\rule[0.5ex]{\linewidth}{1pt} }
Spoofer (in vehicle) moves south along Stavedalsveien (FV7702) from the start while being spoofed to a fixed position at 70 N, 10 E.
\subsubsection*{Power or power range}
Min: 0.1W\\Max: 100W\subsubsection*{Test bands/constellation}
'L1'
\subsubsection*{Transmitter equpment}
'N/A'
\\\subsection{2.6.4  Spoofer (in vehicle) stationary and then moving with first fixed and then moving spoofed position.}

\textcolor{lightgray}{\noindent\rule[0.5ex]{\linewidth}{1pt} }
Spoofer (in vehicle) starts stationary for 10 min, then vehicle begins to drive south along Stavedalsveien (FV7702); spoofed position is approximately true for the first 10 min, then starts to move directly south with constant speed (15 m/s) in a slightly different direction than the vehicle.
\subsubsection*{Power or power range}
Min: 0.1W\\Max: 100W\subsubsection*{Test bands/constellation}
'L1'
\subsubsection*{Transmitter equpment}
'N/A'
\\\section{2.7: Stationary incoherent spoofing with extreme timeshifts (+/- 1 to 2 years)}

\subsection*{Rationale}

Some equipment will use GNSS to synchronize time and this time and different subsystems can use this time for checking validity of licences, certificates etc.This test can be used to check for unintended effects of large time shifts on equipment and subsystems.

\subsection*{Test description}

Providing a date 2 years back in time or 2 years ahead can cause denial of service for certain services.

\subsection*{Additional information}

The effect on subsystems is not known and hence care should be taken to limit the range of the transmission to include only systems that we want to test.

\section*{Test within this testgroup}

\subsection{2.7.1  Pos=True, Time=2 years backwards, Jamming=True, Scenario=Static+motion}

\textcolor{lightgray}{\noindent\rule[0.5ex]{\linewidth}{1pt} }
Time will be shifted by 2 years in the past.
The test will be preceded by jamming (L1, G1, B1l, L2, E5b, L5) 
The jamming will continue during spoofing except on L1/E1
\subsubsection*{Power or power range}
Min: 0.1W\\Max: 100W\subsubsection*{Test bands/constellation}
'L1', 'E1'
\subsubsection*{Transmitter equpment}
'N/A'
\\\subsection{2.7.2  Pos=True, Time=2 years forward, Jamming=True, Scenario=Static+motion}

\textcolor{lightgray}{\noindent\rule[0.5ex]{\linewidth}{1pt} }
Time will be shifted by 2 years in the future.
The test will be preceded by jamming (L1, G1, B1l, L2, E5b, L5) 
The jamming will continue during spoofing except on L1/E1
\subsubsection*{Power or power range}
Min: 0.1W\\Max: 100W\subsubsection*{Test bands/constellation}
'L1', 'E1'
\subsubsection*{Transmitter equpment}
'N/A'
\\\chapter{3 Meaconing}

\section{3.1: Stationary meaconing with varying power and time exposure}

\subsection*{Rationale}

The objective is to observe how equipment and systems behave under meaconing.

\subsection*{Test description}

GNSS re-transmission of real live sky signals, where the GNSS environment will have wrong position with real satellite data, only slightly time delayed.
The test will re-transmitt only (To re-transmitt on other GNSS bands requires an extensive filterbank to exclude all signals outside GNSS frequencies.) the GPS L1 and L2 bands. The re-transmitted signals needs a lot of amplification, with the added risk of amplifying background noise. Therefore, it is hard to give precise estimates of effective power levels and range. Attendees should try to observe PNT changes and/or loss of PNT, and monitor the changes when their equipment and systems are exposed to two different power levels and varying degrees of time exposure to the meaconed signal. Maybe especially interesting is to see if the effects of movement and speed, coupled with other sensor data, will result on the total output. The tests are performed with constant power outputs (0.1 W or 1 W), and with varying lengths of transmission times [see above for power levels]. There are planned a 15-minute break between each test. Many tests will be performed twice, so that it is possible to try to detect differences between stationary and mobile test objects.
The meaconed position is 69.2803484 N, 16.0074695 E.

\subsection*{Additional information}

The jammer employed will be F8.1 "Porcus Major", see appendix A. Power levels denoted in the specific tests below are indications and will only be known during setup the days before Jammertest. Information will be provided during daily pre-test morning briefings.

\section*{Test within this testgroup}

\subsection{3.1.1  0.1 W meaconing}

\textcolor{lightgray}{\noindent\rule[0.5ex]{\linewidth}{1pt} }
0.1 W meaconing
\subsubsection*{Power or power range}
Min: 0.1W\\Max: 0.1W\subsubsection*{Test bands/constellation}
'L1', 'L2'
\subsubsection*{Transmitter equpment}
'F8.1'
\\\subsection{3.1.2  0.1 W meaconing preceded by jamming (20W PRN L1 , L2, L5 and G1)}

\textcolor{lightgray}{\noindent\rule[0.5ex]{\linewidth}{1pt} }
0.1 W meaconing preceded by jamming (20W PRN L1 , L2, L5 and G1)
\subsubsection*{Power or power range}
Min: 0.1W\\Max: 0.1W\subsubsection*{Test bands/constellation}
'L1', 'L2'
\subsubsection*{Transmitter equpment}
'F8.1'
\\\subsection{3.1.3  10 W meaconing}

\textcolor{lightgray}{\noindent\rule[0.5ex]{\linewidth}{1pt} }
10 W meaconing
\subsubsection*{Power or power range}
Min: 10W\\Max: 10W\subsubsection*{Test bands/constellation}
'L1', 'L2'
\subsubsection*{Transmitter equpment}
'F8.1'
\\\subsection{3.1.4  10 W meaconing preceded by jamming (20W PRN L1 , L2, L5 and G1)}

\textcolor{lightgray}{\noindent\rule[0.5ex]{\linewidth}{1pt} }
10 W meaconing preceded by jamming (20W PRN L1 , L2, L5 and G1)
\subsubsection*{Power or power range}
Min: 10W\\Max: 10W\subsubsection*{Test bands/constellation}
'L1', 'L2'
\subsubsection*{Transmitter equpment}
'F8.1'
\\