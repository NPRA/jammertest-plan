\def \filsti {../} %%\def \filsti {./Latex/}
\documentclass[a4paper]{book}

\usepackage{graphicx}
\usepackage{graphbox}
\usepackage{makecell}
\usepackage{multirow}
\usepackage{float}
\usepackage[table,xcdraw]{xcolor}
\graphicspath{ {./Latex} }
\usepackage{tabularray}

% shrink margins:
\usepackage[left=2.5cm, right=3cm, bottom=2.5cm, top=2.5cm]{geometry}

\begin{document}

\chapter{Appendices}
\section{Appendix G - Technical details on jammer equipment}

\subsection{Introduction}
The following section provides technical details on the jammer equipment used in the experiments. The jammers are categorized according to the following scheme:

\begin{table}[H]
\begin{tabular}{|l|l|l|}
\hline \rowcolor[HTML]{C0C0C0} 
\textbf{1st Letter (Norwegian / English)}    & \textbf{1St digit}                             & \textbf{2nd digit}                     \\
\hline
S = Sigarett / Cigarette            & \multirow{4}{*}{Number of antennas}   & \multirow{4}{*}{\# jammer within same category} \\
\cline{1-1}
H = Håndholdt / Handheld            &                                       &                        \\
\cline{1-1}
U = USB / USB stick                 &                                       &                        \\
\cline{1-1}
F = Fastmontert / Permanently installed (Fixed) &                                       &         \\
\hline              
\end{tabular}
\end{table}

\textbf{Exempli gratia:}
S1.2, is a cigarette type jammer, that has 1 antenna, and is unit nr.2 in this category.
\\
\\
\textbf{Additional information:}
\begin{itemize}
    \item Each chapter gives an overview of each jammer brought to Jammertest. As far as possible, it
gives information on
    \begin{itemize}
        \item Centre frequency [MHz]
        \item Bandwidth [MHz]
        \item Power Spectral Density (PSD) [dBm/MHz] for the entire bandwidth
        \item Total output power (TX total) [dBm] for the entire bandwidth
        \item CF max [dBm] (maxhold power at the centre frequency)
        \item Sweep rate [$\mu$s] (if applicable)
        \item Modulation
    \end{itemize}
    \item Indicators such as “L1, L2, L5” etc. are used to indicate main bands of attack, used for
convenience to distinguish between jammers' modus operandi
\item 2023 measurements
    \begin{itemize}
        \item Technical details on low power jammers given in this appendix are from uncalibrated
        measurements. They are rough estimates given for both the frequency and time
        domain. Power levels are not correctly displayed on the chart, because of external
        attenuators used during measurements with a signal analyser. There may also have
        been some constraints in the measurement device, causing fast frequency
        components to not be correctly displayed.
    \end{itemize}
    \item 2024 measurements
    \begin{itemize}
        \item Measurements done with a R\&S FSW. All measurements were performed connected
        directly to the jammers' antenna port, with the other antennas disconnected and (if
        applicable) DIP switches for the other antenna ports disabled. Powe levels etc. should
        be as close to reality as possible for output power at the antenna port.
        \item Throughout the measurements, bandwidth is defined as 3 dB from local (identifiable)
        maxima along the maxhold's descent.
        \item TX power is measured within said bandwidth. Note that TX total is measured over the
        entire bandwidth, so that peak output power is not equal to TX total.
    \end{itemize}
\end{itemize}

%% Content starts here


\input{\filsti equipment}




%% content ends here

\end{document}